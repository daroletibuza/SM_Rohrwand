\chapter{Ergebnisse}
\label{sec:ergebnisse}

\section{Versuchsteil 1: }
\textcolor{red}{Tabellen aufsteigend sortiert nach Druckverlustwerten}

%Tabelle START
\renewcommand{\arraystretch}{1.2}
\begin{table}[h!]
	\centering
	\caption{Druckverluste, Volumenströme und Rohrleitungswiderstände}
	\label{tab:messwerte}
%	\resizebox{12.6cm}{!}{
		\begin{tabular}{c|c|c|c|c}
			\textbf{Messpunkt}	& \textbf{Volumen}& \textbf{Zeit} &\textbf{Druck 1}	& \textbf{Druck 2} \\
			\hline
			\multicolumn{5}{l}{raues Rohr} \\
			\hline
			1&0,01&37,6&0,07&0,01\\
			2&0,01&25,2&0,18&0,04\\
			3&0,01&18,4&0,35&0,10\\
			4&0,01&16,8&0,45&0,14\\
			5&0,01&14,9&0,56&0,17\\
			\hline
			\multicolumn{5}{l}{glattes Rohr} \\
			\hline
			1&0,01&17,1&0,28&0,05\\
			2&0,01&15,0&0,36&0,06\\
			(3)&(0,01)&(12,3)&(0,44)&(0,10)\\
			4&0,01&13,5&0,43&0,07\\
			5&0,01&12,6&0,49&0,09\\
			\hline
			\multicolumn{5}{l}{glattes, dickes Rohr} \\
			\hline
			1&0,01&10,6&0,11&0,04\\
			2&0,01&9,1&0,15&0,07\\
			3&0,01&7,9&0,23&0,10\\
			4&0,01&7,3&0,26&0,12\\
			5&0,01&6,5&0,33&0,16\\
		\end{tabular}
%	}
\end{table}
\FloatBarrier
\vspace*{-2.5mm}
%Tabelle Ende

%Tabelle START
\vspace*{-10.5mm}
\renewcommand{\arraystretch}{1.2}
\begin{table}[h!]
	\centering
	\caption{Druckverluste, Volumenströme und Rohrleitungswiderstände}
	\label{tab:berechnung1}
	%\resizebox{12.6cm}{!}{
		\begin{tabular}{c|c|c|c}
			\textbf{Messpunkt}	& \textbf{Volumenstrom} & \textbf{mittlere Geschwindigkeit}	& \textbf{Druckverlust}\\
			\hline
			\multicolumn{4}{l}{raues Rohr} \\
			\hline
			1&958&1,83&0,06\\
			2&1430&2,73&0,14\\
			3&1962&3,75&0,25\\
			4&2139&4,09&0,32\\
			5&2413&4,61&0,39\\
			\hline
			\multicolumn{4}{l}{glattes Rohr} \\
			\hline
			1&2111&4,04&0,23\\
			2&2406&4,60&0,30\\
			(3)&(2922)&(5,59)&(0,34)\\
			4&2667&5,10&0,36\\
			5&2866&5,48&0,40\\
			\hline
			\multicolumn{4}{l}{glattes, dickes Rohr} \\
			\hline
			1&3403&2,49&0,07\\
			2&3978&2,91&0,08\\
			3&4557&3,33&0,13\\
			4&4932&3,60&0,14\\
			5&5513&4,03&0,17\\
		\end{tabular}
%	}
\end{table}
\FloatBarrier
\vspace*{-2.5mm}
%Tabelle Ende

%Tabelle START
\vspace*{-10.5mm}
\renewcommand{\arraystretch}{1.2}
\begin{table}[h!]
	\centering
	\caption{Druckverluste, Volumenströme und Rohrleitungswiderstände}
	\label{tab:excel}
	%\resizebox{12.6cm}{!}{
	\begin{tabular}{c|c|c|c|c}
		\textbf{Messpunkt}	& \textbf{Temperatur} & \textbf{Dichte}	& \textbf{kine. Viskosität $\left[10^{-7}\right]$} & \textbf{Reynoldszahl}\\
		\hline
		\multicolumn{5}{l}{raues Rohr} \\
		\hline
		1&26,5&996,7&8,66&28766\\
		2&26,8&996,6&8,61&43206\\
		3&27,3&996,5&8,51&59940\\
		4&27,5&996,4&8,47&65639\\
		5&26,3&996,7&8,70&72117\\
		\hline
		\multicolumn{5}{l}{glattes Rohr} \\
		\hline
		1&26,3&996,7&8,70&63108\\
		2&26,1&996,8&8,74&71606\\
		(3)&(25,4)&(997,0)&(8,88)&(85602)\\
		4&25,9&996,9&8,78&78998\\
		5&25,6&996,9&8,84&84344\\
		\hline
		\multicolumn{5}{l}{glattes, dickes Rohr} \\
		\hline
		1&27,3&996,5&8,51&64267\\
		2&27,0&996,6&8,57&74640\\
		3&26,9&996,6&8,59&85318\\
		4&26,6&996,7&8,64&91723\\
		5&26,3&996,7&8,70&101862\\
	\end{tabular}
	%	}
\end{table}
\FloatBarrier
\vspace*{-2.5mm}
%Tabelle Ende

\vspace*{-5mm}
% pgf diagramm3 START
\begin{figure}[h!]
	\begin{center}
		%\resizebox{0.8\textwidth}{!}{
		\begin{tikzpicture}[trim axis left, trim axis right]
		\begin{axis}[
		axis lines = left,
		width = 13cm,
		height = 8cm,
		xmin = 0,
		xmax = 0.45,
		ymin = 0.015,
		ymax = 0.035,
		ylabel={Rohrleitungswiderstand $\lambda$},
		y label style={at={(-0.095,0.5)}},
		xlabel={Druckverlust $\Delta p_V$ in \si{\bar}},
		scaled ticks=false ,
		legend style={at={(1.3,1.1)}}
		]
		%rau
		\addplot[black,mark=*,text mark as node=true,point meta=explicit symbolic,nodes near coords]
		coordinates {(0.056,0.018) (0.14,0.02) (0.25,0.019) (0.315,0.021) (0.382,0.02)};

		%glatt (korrigiert)
		\addplot[black,mark=square*,text mark as node=true,point meta=explicit symbolic,nodes near coords]
		coordinates {(0.235,0.026) (0.3,0.026) (0.358,0.025) (0.405,0.025)};
		
		
		%glatt, dickes Rohr
		\addplot[black,mark=triangle*,text mark as node=true,point meta=explicit symbolic,nodes near coords]
		coordinates {(0.065,0.0309) (0.083,0.0289) (0.125,0.0332) (0.135,0.0306) (0.17,0.0308)};
		
		%Mittelwerte
		\addplot[no markers, dotted] coordinates {(0,0.0196)(0.5,0.0196)};
		\addplot[no markers, dotted] coordinates {(0,0.0254)(0.5,0.0254)};
		\addplot[no markers, dotted] coordinates {(0,0.03089)(0.5,0.03089)};
		
		\legend{raues Rohr, glattes Rohr (korrigiert), glattes/dickes Rohr, Mittelwerte $\lambda$}
		\end{axis}
		\end{tikzpicture}
		%	}
		\caption{$\lambda$ zu Druckverlust }
		\label{dia:sedimentation_1}
	\end{center}
\end{figure}
\FloatBarrier

\vspace*{-5mm}
% pgf diagramm3 START
\begin{figure}[h!]
	\begin{center}
		%\resizebox{0.8\textwidth}{!}{
		\begin{tikzpicture}[trim axis left, trim axis right]
		\begin{axis}[
		axis lines = left,
		width = 15cm,
		height = 8cm,
		xmin = 0,
		xmax = 6000,
		ymin = 0,
		ymax = 0.5,
		ylabel={Druckverlust $\Delta p_V$ in \si{\bar}},
		y label style={at={(-0.095,0.5)}},
		xlabel={Volumenstrom $\dot{V}$ in \si{\raiseto{3} \meter \per \second}},
		scaled ticks=false ,
		legend style={at={(1,1)}}
		]
		%rau
		\addplot[black,mark=*,text mark as node=true,point meta=explicit symbolic,nodes near coords]
		coordinates {(958,0.056) (1430,0.14) (1962,0.25) (2139,0.315) (2413,0.382)};
		
		%glatt (korrigiert)
		\addplot[black,mark=square*,text mark as node=true,point meta=explicit symbolic,nodes near coords]
		coordinates {(2111,0.235) (2406,0.3) (2667,0.358) (2866,0.405)};
		
		
		%glatt, dickes Rohr
		\addplot[black,mark=triangle*,text mark as node=true,point meta=explicit symbolic,nodes near coords]
		coordinates {(3403,0.065) (3978,0.083) (4557,0.125) (4932,0.135) (5513,0.17)};
		
		%Trendlinien
		
		\legend{raues Rohr, glattes Rohr (korrigiert), glattes/dickes Rohr}
		\end{axis}
		\end{tikzpicture}
		%	}
		\caption{Druckverlust zu Volumenstrom}
		\label{dia:V_pV}
	\end{center}
\end{figure}
\FloatBarrier

%Tabelle START
\vspace*{-10.5mm}
\renewcommand{\arraystretch}{1.2}
\begin{table}[h!]
	\centering
	\caption{Druckverluste, Volumenströme und Rohrleitungswiderstände}
	\label{tab:dpla}
	%\resizebox{12.6cm}{!}{
	\begin{tabular}{c|c|c|c}
	\textbf{Messpunkt}	& \textbf{Volumenstrom} &\textbf{Druckverlust}	& \textbf{Rohrleitungswiderstand} \\
	\hline
	\multicolumn{4}{l}{raues Rohr} \\
	\hline
	1&958&0,06&0,018\\
	2&1430&0,14&0,020\\
	3&1962&0,25&0,019\\
	4&2139&0,32&0,021\\
	5&2413&0,41&0,020\\
	\hline
	\multicolumn{4}{l}{glattes Rohr} \\
	\hline
	1&2111&0,24&0,026\\
	2&2406&0,30&0,0,026\\
	(3)&(2922)&(0,034)&(0,020)\\
	4&2667&0,36&0,025\\
	5&2866&0,41&0,025\\
		\hline
	\multicolumn{4}{l}{glattes, dickes Rohr} \\
	\hline
	1&3403&0,07&0,031\\
	2&3978&0,08&0,029\\
	3&4557&0,13&0,033\\
	4&4932&0,14&0,031\\
	5&5513&0,17&0,031\\
	\end{tabular}
%	}
\end{table}
\FloatBarrier
\vspace*{-2.5mm}
%Tabelle Ende

\textcolor{red}{Glatte Rohre müssten weniger Rohrleitungswiderastand haben, da sich aber Dichte des Wasser verringert steigt somit laut Gleichung der Rohrleitungswiderstand}
