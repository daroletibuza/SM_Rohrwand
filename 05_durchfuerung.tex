\chapter{Durchführung}
\label{sec:durchfuerung}

Die Versuchsanlage besteht aus mehreren, einzeln absperrbaren Rohrleitungen unterschiedlicher Durchmesser und teils mit Einbauten. Für den Versuch werden je drei Rohrleitungen – eine angeraute Leitung und zwei hydraulisch glatte Leitungen unterschiedlicher Nenndurchmesser – ohne Einbauten sowie je eine Rohrleitung mit eingebautem Schrägsitzventil und einem Muffenschieber untersucht. \\
Für die rauen und hydraulisch glatten Rohrleitungen werden dazu für je fünf unterschiedliche Wasservolumenströme die Druckverluste in jeder einzelnen Rohrleitung über die Manometer am Ein- und Auslauf ermittelt. \linebreak 
Vorher ist das System zu entlüften. Mittels der Druckverluste und der Strömungsgeschwindigkeiten berechnet sich schließlich für jede Rohrleitung eine entsprechende Rohrreibungszahl $\lambda$. \\
Die Rohrleitungen mit Einbauten werden auf die Druckverlustbeiwerte $\zeta$ untersucht, die durch die jeweiligen eingebauten Armaturen auftreten. Dazu werden die Druckverluste bei einem konstanten Wasservolumenstrom bestimmt, während die Öffnungsweite des Ventils bzw. des Muffenschiebers verändert wird. Die sich daraus ergebende Ventilkennlinie ist als $\zeta$ über den Öffnungswinkel und als kv-Wert über den Ventilhub aufzutragen. \\
Für die Rohrreibungszahl $\lambda$ ist zusätzlich eine Fehlerrechnung durchzuführen, da anzunehmen ist, dass die Messwerte des Versuchs mit Fehlern behaftet sind.\\
Neben der Versuchsanlage mit den Messinstrumenten für Druck und Volumenstrom wurde weiterhin eine Stoppuhr genutzt.